The cascaded controller successfully stabilises around the reference, settling within the quantisation error of the level sensor and remaining there for several hours. We note, however, that the performance of the controller is intimately tied to the KF. The controller cannot settle at the reference until the KF approximately tracks the true consumer demand pattern. Accordingly, the controller must be conservative enough that the error in the disturbance estimate provided by the KF does not induce major oscillations, while being aggressive enough that transients do not fall within the passband of the KF. Failure to find this "Goldilocks" zone turns the filter and controller into a pair of coupled oscillators. This can to some extent be seen after packet loss is introduced; as expected it introduces significant oscillations, and the filter eventually locks onto these rather than the true consumer disturbance pattern.

Control of the pumps in the inner loop remains a difficult issue, with steady-state oscillations around the reference arising regardless of controller parameters. Significant manual tuning is required in the lab as the linearised model is quite inaccurate; practical values of $K_p = 3.5, \ K_i = 1.0$ were identified. We note per \Cref{fig:InnerLoop} that there is very clear coupling between the two pumps, and that this coupling must be lost somewhere in the linearisation process. 

Leakage detection is clearly possible by residual comparison as per \Cref{fig:Leakage}, and we note that lab observations suggest that detection is very robust, remaining possible even if the system is poorly tuned and resultingly oscillatory.

In the future, these results could be made significantly more practical by a formalised approach to controller tuning, or by identifying an alternative disturbance measurement that is less coupled to controller behaviour. Additionally, control in the inner loop could be significantly improved by a MIMO or decoupled SISO approach, although a modelling strategy that captures the pump cross-coupling must first be identified.