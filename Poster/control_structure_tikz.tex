\documentclass{article}
\usepackage[utf8]{inputenc}

\title{Tikz_example}
\author{kasperlaustsen }
\date{October 2021}

% \begin{document}
	% \documentclass{article}
	\usepackage{tikz}
	\usetikzlibrary{shapes,arrows,positioning,calc}
	
	\begin{document}
		
		\tikzset{
			block/.style = {draw, fill=white, rectangle, minimum height=3em, minimum width=3em},
			tmp/.style  = {coordinate}, 
			sum/.style= {draw, fill=white, circle, node distance=1cm},
			input/.style = {coordinate},
			output/.style= {coordinate},
			pinstyle/.style = {pin edge={to-,thin,black}
			}
		}
		
		
		
		\begin{figure}[!htb]
		\centering
		\begin{tikzpicture}[auto, node distance=2.5cm,>=latex']
			% ========================== Nodes ============================
			% Nodes in upper vertical line
			\node [input, name=rinput] (rinput) {};
			\node [sum, right of=rinput,	node distance = 1.5cm] (sum1) {};
			\node [block, right of=sum1, 	node distance = 1.25cm] (LQR) {LQR};
			\node [sum, right of=LQR, 		node distance =	2cm] (sum2) {};
			\node [block, right of=sum2, 	node distance = 1.25cm] (PI){PI};
			\node [block, right of=PI, 		node distance = 2.5cm, align=center] (Fast){Fast\\Dynamics};
			\node [block, right of=Fast, node distance = 3cm, align=center] (Slow){Slow\\Dynamics};
			\node [output, right of=Slow] (output) {};
			
			% Nodes for inner feedback
			\node [tmp, right of=Fast, node distance = 1.5cm] (tmp0){};
			\node [tmp, below of=tmp0, node distance = 1cm] (tmp1){};
			\node [tmp, below of=sum2, node distance = 1cm] (tmp2){};
			
			% Nodes for outer feedback
			\node [tmp, right of=Slow, node distance = 1.5cm] (tmp10){};
			\node [tmp, below of=tmp10,node distance = 1.5cm] (tmp11){};
			\node [tmp, below of=sum1, node distance = 1.5cm] (tmp12){};
			
			% Nodes for Disturbance
			\node [tmp, above of=tmp0, node distance = 1.5cm	] (tmp20){};
			\node [tmp, above of=tmp0, node distance = 1.2cm ] (tmp21){};
			\node [tmp, above of=Fast, node distance = 1.2cm] (tmp22){};
			\node [tmp, above of=Slow, node distance = 1.2cm] (tmp23){};
			
			% ========================== Lines ============================
			
			% Lines in upper vertical part of block diagram
			\draw [->] (rinput) -- node{\begin{tabular}{c} Pressure \\ ref. \end{tabular}} (sum1);
			\draw [->] (sum1) --node[name=z,anchor=north]{} (LQR);
			\draw [->] (LQR) -- node{\begin{tabular}{c} Pump \\ flow \\ ref. \end{tabular}}(sum2);
			\draw [->] (sum2) -- (PI);
			\draw [->] (PI) -- node{\begin{tabular}{c} Pump \\ speed \end{tabular}}(Fast);
			\draw [->] (Fast) -- node{\begin{tabular}{c} Pump \\ flows \end{tabular}}(Slow);
			\draw [->] (Slow) -- node{\begin{tabular}{c} Tank \\ pressure \end{tabular}}(output);	
			
			% Lines for inner feedback
			\draw [-] (tmp0) -- (tmp1);
			\draw [-] (tmp1) -- (tmp2);
			\draw [->] (tmp2) -- node[pos=0.99]{$ - $}(sum2);
			

			% Lines for outer feedback
			\draw [-] (tmp10) -- (tmp11);
			\draw [-] (tmp11) -- (tmp12);
			\draw [->] (tmp12) -- node[pos=0.99]{$ - $}(sum1);

			
			% Lines for disturbance
			\draw [-] (tmp20)node[above, align=center]{Consumer \\ Disturbance} -- (tmp21);
			\draw [-] (tmp21) -- (tmp22);
			\draw [-] (tmp21) -- (tmp23);
%			\draw [->] (tmp22) -- node[pos = 0.5]{\begin{tabular}{c} Valve \\ opening \\degree \end{tabular}}(Fast);
			\draw [->] (tmp22) -- node[pos = 0.5]{}(Fast);
%			\draw [->] (tmp23) -- node[pos = 0.5]{\begin{tabular}{c} Consumer \\ flows \end{tabular}}(Slow);
			\draw [->] (tmp23) -- node[pos = 0.5]{}(Slow);


		\end{tikzpicture}
%		\caption{Control Structure} \label{fig6_10}
		\end{figure}
		
		% \end{document}
	
\end{document}
